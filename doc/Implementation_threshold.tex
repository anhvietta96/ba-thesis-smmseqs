In MMseqs, in order to evaluate a proper threshold \(k\) for the environment, a \(z\)-score statistics was applied. For each query sequence, a calibration search through a subset of 100~000 randomly sampled target sequences is performed, where the number of prefiltering operations
\begin{align}
    \text{sum}_L = \sum_{t=1}^{100~000} (L_t-k+1)
\end{align}
and its sum of scores
\begin{align}
    \text{sum}_S = \sum_{t=1}^{100~000} S_{qt}
\end{align}
are recorded. \(L_t\) here denotes the length of the target sequence \(t\) and \(S_{qt}\) the prefiltering score between query sequence \(s\) and target sequence \(t\). The expected chance prefiltering score between them is then
\begin{align}
    S_0 = (L_t-k+1)\frac{\text{sum}_S}{\text{sum}_L}
\end{align}
Assuming the number of \(q\)-gram matches is Poisson-distributed, the standard deviation of the scores \(\sigma_S\) can be computed through number of expected \(q\)-gram matches \(n_{\text{match}}\):
\begin{align}
    n_{\text{match}} &\approx \frac{S_0}{S_{\text{match}}}
\end{align}
\begin{align}
    \sigma_S &= S_{\text{match}}\sqrt{n_{\text{match}}}\\
    &= \sqrt{(L_t-k+1)\frac{\text{sum}_S}{\text{sum}_L}S_{\text{match}}}
\end{align}
where \(S_{\text{match}}\) is the expected score per chance \(q\)-gram match. The significant prefiltering score \(S_{qt}\) should then fulfill the condition
\begin{align}
    S_{qt} \geq Z_{thr}\sigma_S + S_0
\end{align}
where \(Z_{thr}\) is the significant \(z\)-score. MMseqs2 would take in a sensitivity parameter \(s\), labeled internally as the average length of \(q\)-gram list each sequence position, from the user and using a heuristic to compute for \(S_{qt}\). This approach is shown to be lacking in control for \(s\) and therefore the goal is to streamline the process and to allow for a more fine-grained control of the length of \(q\)-gram list. 

In order to approximate an appropriate threshold \(k\) for a \(q\)-gram environment, a distribution of pairwise amino acid scores is generated. This distribution is then convoluted with itself \(q-1\) times to create a distribution of unsorted-$q$-grams-vs-unsorted-$q$-grams score and the threshold can be then obtained by filtering only a fraction of the top scores. This approach could then be easily adapted for any given target amino acid distribution, where the expected probability of the \(q\)-grams is integrated into the distribution of the pairwise amino acid scores. %But since the evaluting scheme demands a score distribution between sorted and unsorted \(q\)-grams
