With the advent of high-throughput sequencing technologies, the cost of genome sequencing has come down sgnificantly. Sequence databases, such as UniProt have been growing by a factor of two every two years, which leads to a significant focus on developing searching and clustering methods that can handle large-scale datasets efficiently. Algorithms and software that can scale horizontally (across multiple machines) have gained importance.

On sequence searching, new algorithms and heuristics have been developed to improve sensitivity without sacrificing speed. Tools like DIAMOND and MMseqs2 provide fast and sensitive sequence searching capabilities, especially in metagenomics and large-scale sequencing projects. HMM-based methods, such as the HMMER software, have been enhanced to improve their sensitivity and accuracy, making them invaluable for protein domain and family searching. The existing algorithms like BLAST have also seen massive improvement by using GPU parallelization and specific hardwares, namely FPGAs (Field-Programmable Gate Arrays) have been customized for accelerating sequence searching algorithms.

Recent developments on metagenomics have also given ways to specialized databases and algorithms. These databases contain sequences from environmental samples and enable the identification of novel organisms and genes within complex microbial communities.

On sequence clustering, myriads of methods have been developed. graph-based clustering methods, such as Markov Clustering (MCL) and Louvain algorithm, have gained popularity. These methods model sequences as nodes in a graph and use edge weights to represent similarities, allowing for the detection of densely connected clusters within the graph. Density-based methods like DBSCAN (Density-Based Spatial Clustering of Applications with Noise) have been applied in bioinformatics. These algorithms group sequences based on the density of data points, enabling the discovery of clusters with varying shapes and sizes. Traditional distance-based clustering algorithms, such as hierarchical clustering and k-means, have been adapted and optimized for large biological datasets. Efficient distance metrics and clustering strategies have been a focus of research.
\begin{comment}
**Deep Learning in Clustering: Deep learning techniques, particularly autoencoders and neural networks, have been explored for clustering biological sequences. These methods can learn intricate patterns and representations from raw sequence data, potentially improving clustering accuracy.

Recent Software and Tools:
**CD-HIT: CD-HIT is a widely used tool for clustering biological sequences. It allows users to cluster large datasets and select a representative sequence from each cluster based on a specified threshold.

**UCLUST: UCLUST is part of the USEARCH suite and is designed for fast sequence clustering. It uses a greedy algorithm to form clusters and is efficient for large datasets.

**MMseqs2: MMseqs2 is a software suite for sequence searching and clustering. It provides fast and sensitive sequence clustering algorithms, making it suitable for large-scale bioinformatics analyses.

**DADA2: DADA2 is a bioinformatics pipeline for the analysis of amplicon sequencing data (e.g., 16S rRNA). While it's primarily used for sequence error correction, it also involves the clustering of similar sequences into amplicon sequence variants (ASVs).

**Swarm: Swarm is a density-based clustering algorithm designed to cluster biological sequences. It's particularly useful for clustering operational taxonomic units (OTUs) in microbial ecology studies.

**HDBSCAN: HDBSCAN (Hierarchical Density-Based Spatial Clustering of Applications with Noise) is a density-based clustering algorithm that can find clusters of varying densities. It's useful for discovering clusters in datasets where clusters of different densities exist.
\end{comment}