Ungapped Alignment in MMseqs2:
Prefiltered Candidates:
After the prefiltering stage, MMseqs2 has a reduced set of candidate sequences that are likely to contain significant matches to the query sequence. These sequences are the input for the ungapped alignment stage.

Seed-Based Matching:
Ungapped alignment often starts with seed-based matching. Seeds are short, contiguous sequences (k-mers) that are used as anchors to identify potential alignment positions in the candidate sequences. MMseqs2 uses these seeds to rapidly find exact matches or approximate matches within a certain tolerance (such as allowing a limited number of mismatches).

Seed Extension:
Once seeds are identified, MMseqs2 extends these seeds in both directions to form ungapped alignments. The extensions are performed by verifying the surrounding residues of the seeds to confirm the match quality. Sequences with extensions meeting the specified criteria are retained as ungapped alignments.

Score Calculation:
Ungapped alignments are usually scored based on the number of matching residues, the positions of these matches, and the quality of the matches (e.g., matches in conserved regions might be given higher scores). The scoring system ensures that only high-quality, significant matches are considered for further analysis.

E-value Calculation:
MMseqs2 might calculate an E-value for each ungapped alignment. The E-value represents the expected number of alignments of similar quality that would occur by chance in a database of a particular size. Lower E-values indicate more significant alignments.

Output:
The output of the ungapped alignment stage includes the sequences that have passed the seed-based matching and extension steps. These sequences are presented as hits or alignments. Each hit typically includes information such as the query sequence, the target sequence, the alignment position, the alignment score, and the E-value.

Further Processing:
The ungapped alignments obtained from this stage might be used as input for subsequent stages in the analysis pipeline, such as gapped alignments or clustering, depending on the specific analysis or task being performed.

