



\begin{comment}
Widely used substitution matrix in bioinformatics for scoring the similarity between amino acid sequences, specifically in sequence alignment algorithms like BLAST (Basic Local Alignment Search Tool) and PSI-BLAST, BLOSUM stands for "BLOcks SUbstitution Matrix" and is derived from comparing conserved sequence blocks in related proteins. The relatedness of these proteins is often quantified as sequence identity, or the proportion of identical residues between two sequences when aligned. Depends on the context, this identity might vary, resulting in different BLOSUM score matrices, mostly used are BLOSUM80, BLOSUM62 or BLOSUM45, which correspond to 80~\%, 62~\% and 45~\% sequence identity respectively.

In order to obtain a BLOSUM score matrix, a database of sequences is first filtered to the sequence identity, and the sequences over the identity threshold are removed. From the remaining sequences,  pairwise amino acid substitutions in each column are counted and normalized to frequency. The score between a pair of amino acids is then computed as the log-odds ratio of observing a given amino acid substitution in the conserved blocks compared to what would be expected by chance.

An alternative to BLOSUM scoring scheme is the PAM (Point Accepted Mutation) matrices, which are based on the concept of evolutionary distances and represent the probabilities of specific amino acid substitutions over a certain period of evolutionary time. They are derived underthe assumption that the sequences being compared have evolved under a process of constant evolutionary rates. Similar to BLOSUM, PAM models are often based on observed differences in closely related protein sequences.

The first step in creating a PAM matrix involves calculating the evolutionary distance, which represents the average number of substitutions per amino acid site that has occurred over a specific evolutionary period. This distance is often denoted as PAM1, indicating one substitution per 100 amino acids.  PAM1 is then extrapolated to create matrices for higher evolutionary distances, such as PAM250. The extrapolation assumes that the process of amino acid substitution is stationary and reversible, meaning that the substitution rates remain constant over time and substitutions can occur in both directions. The scores in a PAM matrix represent the logarithm of the odds ratio of observing a particular amino acid substitution compared to what would be expected by chance, given the assumed evolutionary model.

PAM matrices are used in sequence alignment algorithms, similar to BLOSUM matrices. Specifically, they are employed in tools like BLAST and Needleman-Wunsch algorithm for database searches and pairwise sequence alignments, respectively.
\end{comment}