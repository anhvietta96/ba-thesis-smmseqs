In order to quantify the similarity between sequences, a scoring function is often used. They provide quantitative measures for comparing biological sequences, structures, and interactions, facilitating a deeper understanding of biological processes and aiding in drug discovery, evolutionary analysis, and functional genomics research. In sequence alignment context, these functions often compare pairwise the biological residues (often nucleotide bases or proteinogenic amino acids). The most common scoring function is the substitution matrix, often represented as a matrix of scores indicating the likelihood of one amino acid (or nucleotide) being replaced by another. The BLOSUM (Blocks Substitution Matrix) and PAM (Point Accepted Mutation) matrices are examples of substitution matrices widely used in bioinformatics. The two families of score matrices can be summarized as a mathematical function:

\begin{align}
\sigma: \mathcal{A} \times \mathcal{A} \rightarrow \mathcal{R}
\end{align}

 A positive score between two residues often denotes similarity between them and a negative score indicates dissimilarity. The score of between two sequences \(u,v\in \mathcal{A}^q\) can then be defined as follow:

\begin{align}
\sigma(u,v)=\sum_{i=0}^{|u|-1}\sigma(\Subchar{u}{i},\Subchar{v}{i})\label{equation:sequenceScore}
\end{align}

In order to compute the $k$-environment for a \(q\)-gram \(u\) efficiently, MMseqs2 decomposes it to groups of sub-\(q\)-grams, specifically trigrams and digrams \(u_i\), prioritizing trigrams. For \(q\in \{2,3\}\), a score matrix \(\Scoretable{u}_q=\lbrack (v,\sigma(u,v))\mid v\in\Alpha^{q}\rbrack\) is computed and sorted after the subscores \(\sigma(u,v))\).

The full $k$-environment of a \(q\)-gram \(u\) can then be iterated as a subset of the cartesian product \(\bigtimes ST(u_i)\):
\begin{align}
\text{Env}_k(u) = \{(v,\sigma(u,v))|(v,\sigma(u,v))\in \bigtimes ST(u_i),\sigma(u,v)\geq k\}\label{ScoreTablesCartesian}
\end{align}
The iteration over primary environments can then stop, when it's no longer feasible to reach the threshold \(k\).

The key to improve over the method outlined in MMseqs2, is the following:

Given a bijective mapping \(\varphi:\{0,1,\ldots,q-1\}\to \{0,1,\ldots,q-1\}\). For any sequence of length \(q\),  we define a function
\begin{align}
\varphi(u)=\Subchar{u}{\varphi(0)}\Subchar{u}{\varphi(1)}\ldots\Subchar{u}{\varphi(q-1)}  
\end{align}
in that, \ \(\varphi\) permutate the residue in \(u\).

Given a special permutation \(\Permname{u}\), so that it orders the characters in \(u\): \(\Subchar{u}{\Perm{u}{i}}\leq\Subchar{u}{\Perm{u}{i+1}}\forall i, 0\leq i\leq q-2\). The resulted \(q\)-gram \(u_s\) from the permutation is called sorted \(q\)-gram and could be rearranged using the inverse function:
\begin{align}
\Perminverse{u}{\Perm{u}{i}}=i \forall i, 0\leq i\leq q-1 \Leftrightarrow \Perminverse{u}{\Perm{u}{u}}=u
\end{align}

An observation is, that given a \(q\)-gram \(u\), a permutation \(\varphi_u\) and the mapping \(u[i] = (\varphi_u (u))[j]\)
\begin{align}
\sigma(\Perm{u}{u},\Perm{u}{v}) &= \sum_{j=0}^{q-1} \sigma ((\varphi_u (u))[j], (\varphi_u (v))[j])\\  
&= \sum_{i=0}^{q-1} \sigma (u[i],v[i]) \\
&= \sigma(u,v)
\end{align}
which means, the same permutation function applied on both sequences \(u\) and \(v\)
permutates the characters in the same way and the score between the unpermutated sequences is equal to the score between the permutated. This means that the score matrix ST can be computed only for the sorted \(q\)-grams, and it would hold all scores needed to compute the full $k$-environment.

The advantage of this approach is the smaller number of sorted $q$-grams compared to unsorted, leading to a more efficient memory usage and a more compacted computation of the score matrix ST. Any overhead caused by the rearragement of the sorted $q$-grams can be reduced by SIMD instructions.