As compile constants, the program takes in a class $ScoreClass$, which represents a scoring function and a spaced seed. 

ScoreClass should contain a C-style $char$ array called $character\_spec$ detailing all bijective mappings from $\mathcal{A}$ to $[0,|\mathcal{A}|-1]$ and a 64-bit numeric value $num\_of\_chars$ equal to $|\mathcal{A}|$. Additionally the class should contain a C-style array of size $(num\_of\_chars)^2$ called $score\_matrix$ describing $f(x,y) \forall x,y\in \mathcal{A}$. The minimum and maximum value in $ScoreClass$ should also be included as $smallest\_score$ and $highest\_score$ respectively.

The spaced seed is represented as a numeric 64-bit constant, which in compiling will be converted into a bitset.

At run time the program takes in a sequence in FASTA format. Any data error, for example empty file or wrong data format will automatically lead to termination of the program and an error message will be logged. These behaviours are well-defined in class GttlMultiseq.

Additionally some fixed parameters, for example maximum sub-$q$-gram length and acceptance rate are defined as preprocessor constants in $utils.hpp$